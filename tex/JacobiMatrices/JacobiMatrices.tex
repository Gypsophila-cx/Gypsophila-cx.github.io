\documentclass[a4paper, 10pt]{article}
%引用设置使用Bibtex
\usepackage{cite}
%\usepackage{gbt7714}
%\bibliographystyle{gbt7714-numerical}
%页面设置
\usepackage{geometry}
%字体设置
\usepackage{fontspec}
%\setmainfont{Times New Roman}
%定理环境
\usepackage{amsmath}
\numberwithin{equation}{section}
\usepackage{amsthm}
\newtheorem*{definition}{Definition}
\newtheorem{theorem}{Theorem}
\newtheorem{lemma}{Lemma}
\newtheorem*{corollary}{Corollary}
\newtheorem*{proposition}{Proposition}
\newtheorem*{example}{Example}
%数学环境字体
\usepackage{bm}
\usepackage[all]{xy}
%加载 TikZ 用于绘制交换图
\usepackage{tikz-cd}
%颜色
\usepackage{color,xcolor}
\usepackage{booktabs}

\definecolor{miku}{RGB}{57,197,187}
\definecolor{sakura}{RGB}{255,192,203}
\definecolor{rose}{RGB}{255,228,225}
\definecolor{brown}{RGB}{210,105,30}
\definecolor{lbrown}{RGB}{239,235,224}
\definecolor{bule}{RGB}{0,47,167}
\definecolor{lyellow}{RGB}{250,250,210}
\definecolor{lpurple}{RGB}{255,240,245}
\definecolor{lbule}{RGB}{135,206,250}
\definecolor{gbule}{RGB}{64,224,208}
\definecolor{green}{RGB}{138,200,207}
\definecolor{lgreen}{RGB}{225,255,255}
\definecolor{lorange}{RGB}{248,172,140}
\definecolor{salmon}{RGB}{250,128,114}
\definecolor{burgundy}{rgb}{0.5, 0.0, 0.13}
%链接设置
\usepackage[colorlinks=true,pdfstartview=FitH,linkcolor=blue,anchorcolor=violet, citecolor=magenta]{hyperref} 
%封面
\usepackage{pdfpages}
\usepackage{mathrsfs}
\usepackage{amssymb}
\usepackage{graphicx}
\usepackage{lipsum}
%彩色框
\usepackage{framed}
\usepackage{tcolorbox}
\tcbuselibrary{breakable}
\tcbuselibrary{theorems}
\tcbuselibrary{skins}
\usepackage{colortbl}
\usepackage{float}
\usepackage[export]{adjustbox}
\newtcolorbox[auto counter,number within=section]{notebox}[2][]{%
colback=miku!2!white,
colframe=miku,
coltitle=white,
fonttitle=\bfseries,
rightrule=2pt,
leftrule=2pt,
bottomrule=2pt,
colbacktitle=miku,
theorem style=standard,
breakable,
arc=2pt,
drop fuzzy shadow=black!20!white,
title=Note~\thetcbcounter: #2,#1}
\newtcolorbox[auto counter,number within=section]{markbox}[2][]{%
colback=miku!2!white,
colframe=miku,
coltitle=white,
fonttitle=\bfseries,
rightrule=0pt,
leftrule=0pt,
bottomrule=2pt,
colbacktitle=miku,
theorem style=standard,
breakable,
arc=0pt,
drop fuzzy shadow=black!20!white,
title=Remark~\thetcbcounter: #2,#1}
\newtcolorbox[no counter]{theorems}[2][]{%
width=12cm,
center,
sidebyside,
sidebyside adapt=left,
sidebyside gap=6mm,
sidebyside align=center seam,
colback=burgundy!2!white,
colframe=burgundy,
coltitle=white,
fonttitle=\bfseries,
rightrule=1pt,
leftrule=1pt,
bottomrule=2pt,
colbacktitle=burgundy,
theorem style=standard,
enhanced,
drop fuzzy shadow southeast=black!30!white,
breakable,
arc=0pt,
title=Theorem. #2,#1}
\newtcolorbox[no counter]{definitions}[2][]{%
width=12cm,
center,
colback=lyellow!2!white,
colframe=yellow!3!lyellow,
coltitle=bule,
fonttitle=\bfseries,
rightrule=0pt,
leftrule=1pt,
bottomrule=2pt,
colbacktitle=lyellow,
theorem style=standard,
breakable,
arc=5pt,
enhanced,
drop fuzzy shadow southeast=black!20!white,
title=Definition. #2,#1}
\newtcolorbox[auto counter,number within=section]{corollarys}[2][]{%
colback=lyellow!2!white,
colframe=lyellow,
coltitle=bule,
fonttitle=\bfseries,
rightrule=0pt,
leftrule=1pt,
bottomrule=2pt,
colbacktitle=lyellow,
theorem style=standard,
breakable,
arc=0pt,
enhanced,
drop fuzzy shadow southeast=black!20!white,
title=Corollary~\thetcbcounter: #2,#1}
\newtcolorbox[auto counter,number within=section]{lemmas}[2][]{%
width=12cm,
center,
colback=lyellow!2!white,
colframe=lorange!30!sakura,
coltitle=bule,
fonttitle=\bfseries,
rightrule=0pt,
leftrule=1pt,
bottomrule=2pt,
colbacktitle=lorange!30!sakura,
theorem style=standard,
breakable,
arc=5pt,
enhanced,
drop fuzzy shadow southeast=black!20!white,
title=Lemma. #2,#1}
\newtcolorbox[auto counter,number within=section]{propositions}[2][]{%
width=12cm,
center,
colback=salmon!5,
colframe=salmon!90!black,
coltitle=white,
fonttitle=\bfseries,
rightrule=1pt,
leftrule=1pt,
bottomrule=2pt,
colbacktitle=salmon!90!black,
theorem style=standard,
breakable,
arc=5pt,
enhanced,
drop fuzzy shadow southeast=black!20!white,
title=Proposition. #2,#1}
\newtcolorbox[no counter]{egbox}[2][]{%
width=12cm,
center,
colback=black!5!white,
colframe=black!20!white,
coltitle=black,
fonttitle=\bfseries,
rightrule=1pt,
leftrule=1pt,
bottomrule=2pt,
colbacktitle=black!20!white,
theorem style=standard,
breakable,
arc=0pt,
enhanced,
drop fuzzy shadow southeast=black!20!white,
title=Example. #2,#1}

%\begin{figure}[H]
%\centering
%\includegraphics[center]{pic.png}
%\end{figure}
\geometry{left=3cm,right=3cm,top=2cm,bottom=2cm}
\tcbuselibrary{most}

\usepackage{algorithm}
\usepackage{algorithmicx}
\usepackage{algpseudocode}

\usepackage{holtpolt}

%自定义设置
\renewcommand{\proofname}{Proof.}
\renewcommand{\contentsname}{ Content }
\newcommand{\image}[2]{
    \centering
    \includegraphics[width={#1}\textwidth]{#2}
}



\newcommand\keywords[1]{\vskip2ex\par\noindent\normalfont{\textbf{关键词}: #1}}
\newcommand{\ekeywords}[1]{\vskip2ex\par\noindent\normalfont{\bfseries Key Words: }#1}
\newcommand{\miku}{\textcolor{miku}}
\newcommand{\sakura}{\textcolor{sakura}}
\newcommand{\brown}{\textcolor{brow}}
\newcommand{\red}{\textcolor{red}}
\newcommand{\blue}{\textcolor{blue}}
\newcommand{\A}{\mathcal{A}}
\newcommand{\C}{\mathbb{C}}
\newcommand{\al}{\alpha}
\newcommand{\sa}{$\sigma$-algebra}
\newcommand{\Bsa}{Borel $\sigma$-algebra}
\newcommand{\F}{\mathcal{F}}
\newcommand{\N}{\mathcal{N}}
\newcommand{\M}{\mathcal{M}}
\newcommand{\m}{ $\mathcal{M}$ }
\newcommand{\B}{\mathcal{B}}
\newcommand{\myP}{\mathcal{P}}
\renewcommand{\bf}[1]{\textbf{#1}}
\newcommand{\dd}{\mathrm{d}}

\newcommand{\myRom}[1]{\uppercase\expandafter{\romannumeral#1}}
\newcommand{\pl}{$ L^p(X) $}
\newcommand{\twol}{$ L^2(X) $}


\begin{document}
\hfill\vbox{\hbox{Numerical Analysis}\hbox{Chen Xi}\hbox{Summer, 2025}}

\begin{center}\Large
    \textbf{Jacobi Matrices}
\end{center}

\section{Jacobi Matrix}
A \emph{Jacobi matrix} is a symmetric tridiagonal matrix of the form:
\begin{equation}\label{eq:jacobi-star}
T_n =
\begin{bmatrix}
\gamma_1 & \delta_2 &        &        &       \\
\delta_2 & \gamma_2 & \delta_3 &        &       \\
         & \ddots  & \ddots   & \ddots &       \\
         &         & \delta_{n-1} & \gamma_{n-1} & \delta_n \\
         &         &        & \delta_n & \gamma_n
\end{bmatrix}.
\tag{$*$}
\end{equation}

More generally, a \emph{Jacobi operator} is a symmetric linear operator on sequences, represented by an infinite tridiagonal matrix. It specifies systems of orthonormal polynomials with respect to a finite, positive Borel measure. The name originates from Jacobi's 1848 theorem stating that every symmetric matrix over a principal ideal domain is congruent to a tridiagonal matrix.

\section{Algebraic Properties}

Let $\{\varphi_n\}_{n=0}^\infty$ be orthonormal polynomials satisfying the three-term recurrence
\begin{equation}\label{eq:three-term-varphi}
\delta_{n+1}\varphi_{n}(\lambda) = (\lambda - \gamma_n)\varphi_{n-1}(\lambda) - \delta_n\varphi_{n-2}(\lambda).
\end{equation}
Then the finite Jacobi matrix $T_n$ is as in \eqref{eq:jacobi-star}.

A simple calculation shows that the characteristic polynomial satisfies:
\begin{equation}\label{eq:char-recursion}
\det(\lambda I - T_n) = (\lambda - \gamma_n)\det(\lambda I - T_{n-1}) - \delta_n^2\det(\lambda I - T_{n-2}).
\end{equation}
Defining $\psi_n(\lambda):=\det(\lambda I - T_n)$, \eqref{eq:char-recursion} becomes
\begin{equation}\label{eq:psi-recursion}
\psi_n(\lambda) = (\lambda - \gamma_n)\psi_{n-1}(\lambda) - \delta_n^2\psi_{n-2}(\lambda).
\end{equation}
This recurrence corresponds to another tridiagonal matrix
\begin{equation}
\hat{T}_n =
\begin{bmatrix}
\gamma_1 & 1        &        &        &       \\
\delta_2^2 & \gamma_2 & 1      &        &       \\
           & \ddots  & \ddots & \ddots &       \\
           &         & \delta_{n-1}^2 & \gamma_{n-1} & 1 \\
           &         &        & \delta_n^2 & \gamma_n
\end{bmatrix}.
\end{equation}
In fact,
\[
\psi_n(\lambda) = \Bigl(\prod_{j=2}^{n+1}\delta_j\Bigr)\varphi_n(\lambda),
\]
so $\psi_n$ is the monic version of $\varphi_n$. Consequently, the eigenvalues of $T_n$ are the roots of both $\psi_n(\lambda)$ and $\varphi_n(\lambda)$.

\subsection{Lanczos Algorithm}
 
The \emph{Lanczos algorithm} is the special version of the \emph{Arnoldi algorithm} applied to symmetric matrices, which is used to compute a orthonormal basis of the Krylov subspace $\mathcal{K}_n(A, v) = \operatorname{span}\{v, Av, A^2v, \ldots, A^{n-1}v\}$. 

\begin{algorithm}[H]
    \caption{Arnoldi Algorithm}
    \label{alg:arnoldi}
    \begin{algorithmic}[1]
        \Statex \textbf{Input:}  Matrix $A\in\mathbb{R}^{m\times m}$, vector $v\in\mathbb{R}^m$, integer $n$.
        \Statex \textbf{Output:} Orthonormal basis $\{q_1, q_2, \ldots, q_n\}$ of $\mathcal{K}_n(A, v)$ and upper Hessenberg matrix $H_n$.
        \State Normalize $q_1 = \frac{v}{\|v\|}$.
    \end{algorithmic}
\end{algorithm}

\section{Multiplication Matrix}
The \emph{multiplication matrix} for Chebyshev polynomials is defined by
\begin{equation}
M_x =
\begin{bmatrix}
0 & \tfrac12 &        &        &        &       \\
1 & 0 & \tfrac12 &        &        &       \\
  & \tfrac12 & 0 & \tfrac12 &        &       \\
  &        & \ddots & \ddots & \ddots &       \\
  &        &        & \tfrac12 & 0 & \tfrac12 \\
  &        &        &        & \tfrac12 & 0
\end{bmatrix},
\end{equation}
which is \textbf{not} symmetric and thus not a Jacobi matrix. However, $M_x$ and the Jacobi matrix share eigen-structure: the eigenvalues of $M_x$ are the first-kind Chebyshev points,
\[
x_k = \cos\bigl(\tfrac{(2k-1)\pi}{2n}\bigr),\quad k=1,2,\dots,n,
\]
and the eigen-decomposition is
\[
M_x V = V X,
\]
with
\[
X = \operatorname{diag}(x_1, x_2, \dots, x_n),
\]
\[
V =
\begin{bmatrix}
1      &        &        &        \\
       & 2      &        &        \\
       &        & 2      &        \\
       &        &        & \ddots\\
       &        &        &        & 2
\end{bmatrix}
\begin{bmatrix}
T_0(x_1) & T_0(x_2) & \cdots & T_0(x_n) \\
T_1(x_1) & T_1(x_2) & \cdots & T_1(x_n) \\
\vdots   & \vdots   & \ddots & \vdots   \\
T_{n-1}(x_1) & T_{n-1}(x_2) & \cdots & T_{n-1}(x_n)
\end{bmatrix}.
\]
One verifies that if $f_1(x)=T_0(x)=1$, then the relations
\[
\begin{aligned}
 f_2(x_j)&=2x_jf_1(x_j),\\
 f_3(x_j)&=2x_jf_2(x_j)-2f_1(x_j),\\
 f_{k+1}(x_j)&=2x_jf_k(x_j)-f_{k-1}(x_j),\quad k=3,\dots,n-1,
\end{aligned}
\]
yield $f_{k+1}(x)=2T_k(x)=2\cos(k\arccos(x))$ for $k\ge1$, matching the Chebyshev definition.

Thus the matrix–vector product $b=Vc$ has entries
\[
b_k=(2-\delta_{1k})\sum_{j=1}^n c_j\cos\bigl(k\tfrac{(2j-1)\pi}{2n}\bigr),\quad \delta_{1k}=\begin{cases}1,&k=1,\\0,&k\neq1,\end{cases}
\]
which is a DCT-II computation and can be performed in $\mathcal{O}(n\log n)$ time.

\end{document}